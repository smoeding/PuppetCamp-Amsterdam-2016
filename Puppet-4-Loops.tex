\documentclass[english,aspectratio=43,nohandout]{beamer}
\usetheme{Portland}

\usepackage[utf8]{inputenc}
\usepackage[T1]{fontenc}
\usepackage{babel}
\usepackage{bera}

\title{Iteration \& Loops in Puppet 4}
\subtitle{Repeating yourself using the Future Parser}
\date{Puppet Camp Amsterdam\\February 5th, 2016}
\author{Stefan Möding}
\titlegraphic{\includegraphics[scale=0.15]{gfx/pclogo}}

% https://ttboj.wordpress.com/2013/11/17/iteration-in-puppet/
% https://tobrunet.ch/2013/01/iterate-over-datastructures-in-puppet-manifests/
% https://www.devco.net/archives/2015/12/16/iterating-in-puppet.php
% https://docs.puppetlabs.com/puppet/latest/reference/lang_iteration.html

\usepackage{textcomp}
\usepackage{booktabs}
\usepackage{appendixnumberbeamer}

%%%%%%%%%%%%%%%%%%%%%%%%%%%%%%%%%%%%%%%%%%%%%%%%%%%%%%%%%%%%%%%%%%%%%%%%%%%%%%
\usepackage{tikz}
\usetikzlibrary{backgrounds,shadows,shapes}

\tikzstyle{background grid}=[draw,black!20,step=.25cm]
\tikzstyle{redmark}=[puppetorange,line width=1pt]
\tikzstyle{nopadding}=[inner sep=0pt]
\tikzstyle{every picture}+=[remember picture]

% http://tex.stackexchange.com/questions/84143/fancy-arrows-with-tikz
\tikzfading[
  name=arrowfading,
  top color=transparent!0,
  bottom color=transparent!30
]

\newcommand*{\tikdownarrow}[1]{%
  \tikz[
    baseline=(A.base),
    font=\Huge\sffamily
  ]
  \node[
    single arrow,
    single arrow head extend=10pt,
    single arrow tip angle=130,
    shape border rotate=270,
    draw,
    inner sep=8pt,
    top color=white,
    bottom color=puppetorange,
    general shadow={fill=black,shadow yshift=-3pt,path fading=arrowfading}
  ] (A) {#1};%
}

%%%%%%%%%%%%%%%%%%%%%%%%%%%%%%%%%%%%%%%%%%%%%%%%%%%%%%%%%%%%%%%%%%%%%%%%%%%%%%
\usepackage{listings}

\lstdefinelanguage{Puppet}{%
  morekeywords=[1]{and, application, attr, case, class, consumes, default,
    define, else, elsif, environment, false, function, if, import, in,
    inherits, node, or, private, produces, true, type, undef, unless,
    augeas, cron, exec, file, filebucket, group, host, interface, k5login,
    macauthorization, mailalias, maillist, mcx, mount, notify, package,
    resources, router, schedule, scheduled_task, selboolean, selmodule
    service, ssh_authorized_key, sshkey, stage, tidy, user, vlan, yumrepo,
    zfs, zone, zpool},
  morekeywords=[2]{alert, assert_type, contain, create_resources, crit,
    debug, defined, digest, each, emerg, epp, err, fail, file, filter,
    fqdn_rand, generate, hiera, hiera_array, hiera_hash, hiera_, nclude,
    include, info, inline_epp, inline_template, lookup, map, match, md5,
    notice, real, ze, reduce, regsubst, require, scanf, sha1, shellquote,
    slice, split, sprintf, tag, tagged, template, versioncmp, warning, with},
  morekeywords=[3]{abs, any2array, base64, basename, bool2num, capitalize,
    ceiling, chomp, chop, concat, convert_base, count, defined_with_params,
    delete, delete_at, delete_values, delete_undef_values, difference,
    dirname, dos2unix, downcase, empty, ensure_packages, ensure_resource,
    flatten, floor, fqdn_rand_string, fqdn_rotate, get_module_path, getparam,
    getvar, grep, has_interface_with, has_ip_address, has_ip_network,
    has_key, hash, intersection, is_array, is_bool, is_domain_name, is_float,
    is_function_available, is_hash, is_integer, is_ip_address,
    is_mac_address, is_numeric, is_string, join, join_keys_to_values, keys,
    loadyaml, load_module_metadata, lstrip, max, member, merge, min,
    num2bool, parsejson, parseyaml, pick, prefix, assert_private, pw_hash,
    range, reject, reverse, rstrip, shuffle, size, sort, squeeze, str2bool,
    str2saltedsha512, strftime, strip, suffix, swapcase, time, to_bytes,
    try_get_value, type3x, type_of, union, unique, unix2dos, upcase,
    uriescape, validate_absolute_path, validate_array, validate_augeas,
    validate_bool, validate_cmd, validate_hash, validate_integer,
    validate_numeric, validate_re, validate_slength, validate_string, values,
    values_at, zip, file_line},
  sensitive,%
  morecomment=[l]\#,%
  morestring=[b]",%
  morestring=[b]',%
}[keywords,comments,strings]

\lstset{%
  language=Puppet,%
  basicstyle=\ttfamily\small,%
  frame=none,%
  showstringspaces=false,%
  identifierstyle=\color{black},%
  keywordstyle={[1]\color{puppetblue}\textbf},%
  keywordstyle={[2]\color{puppetblue}\textbf},%
  keywordstyle={[3]\color{puppetblue}\textbf},%
  stringstyle=\color{puppetviolet2},%
  commentstyle=\itshape\color{puppetorange},%
  emptylines=1,%
  aboveskip=2ex,%
  belowskip=2ex,%
  captionpos=b,%
  breaklines=true,%
  breakatwhitespace=true,%
}

\usepackage{array}
\usepackage{graphicx,color}
\graphicspath{ {./gfx/} }

\newcommand{\iconentry}[2]{\includegraphics[scale=0.15]{#1} & #2\tabularnewline}
\newcommand{\mailto}[1]{\href{mailto:#1}{\nolinkurl{#1}}}

% Disable footnote numbers
\renewcommand{\thefootnote}{\relax}

%%%%%%%%%%%%%%%%%%%%%%%%%%%%%%%%%%%%%%%%%%%%%%%%%%%%%%%%%%%%%%%%%%%%%%%%%%%%%%
\begin{document}

\begin{frame}
\titlepage
\end{frame}

%%%%%%%%%%%%%%%%%%%%%%%%%%%%%%%%%%%%%%%%%%%%%%%%%%%%%%%%%%%%%%%%%%%%%%%%%%%%%%
\section{Introduction}
\subsection{Does Puppet need loops?}

\begin{frame}{\insertsection}{\insertsubsection}

Loops in Puppet?\pause
\begin{itemize}
\item Puppet describes a state to enforce
\item A resource can't be declared more than once
\item So why would we need loops?
\end{itemize}

\vspace{\baselineskip}\pause

Use case: create resources for each array/hash item
\begin{itemize}
\item Avoid copy/paste manifests
\end{itemize}

\vspace{\baselineskip}\pause

\begin{block}{Purpose}
Iterate an array or hash and do something for each element
\end{block}
\end{frame}

%%%%%%%%%%%%%%%%%%%%%%%%%%%%%%%%%%%%%%%%%%%%%%%%%%%%%%%%%%%%%%%%%%%%%%%%%%%%%%
\section{Evolution of Puppet}
\subsection{Puppet 3}

\begin{frame}{\insertsection}{\insertsubsection}

Loops in Puppet 3
\begin{itemize}
\item Essentially no loop statements
\end{itemize}

\vspace{\baselineskip}\pause

Workarounds
\begin{itemize}
\item Array of strings as resource title
\item \texttt{create\_resources} with a parameter hash
\item Recursion using a defined type
\end{itemize}
\end{frame}

%%%%%%%%%%%%%%%%%%%%%%%%%%%%%%%%%%%%%%%%%%%%%%%%%%%%%%%%%%%%%%%%%%%%%%%%%%%%%%
\subsection{Puppet 3 -- Example: array as resource title}

\begin{frame}{\insertsection}{\insertsubsection}
\lstinputlisting{lst/arrayparams.pp}

\vspace{\baselineskip}\pause

Limitations
\begin{itemize}
\item Resource parameters affect all resources
\item Defined type required for more challenging tasks
\end{itemize}
\end{frame}

%%%%%%%%%%%%%%%%%%%%%%%%%%%%%%%%%%%%%%%%%%%%%%%%%%%%%%%%%%%%%%%%%%%%%%%%%%%%%%
\subsection{Puppet 3 -- Example: \texttt{create\_resources}}

\begin{frame}{\insertsection}{\insertsubsection}

\texttt{create\_resources}
\begin{itemize}
\item Indirect way to create resources
\item Uses type name and hash of type title \& parameters
\end{itemize}

\vspace{\baselineskip}\pause

\lstinputlisting{lst/create_resources-1.pp}\pause
\lstinputlisting{lst/create_resources-2.pp}
\end{frame}

%%%%%%%%%%%%%%%%%%%%%%%%%%%%%%%%%%%%%%%%%%%%%%%%%%%%%%%%%%%%%%%%%%%%%%%%%%%%%%
\begin{frame}{\insertsection}{\insertsubsection}
\lstinputlisting[basicstyle=\tiny]{lst/create_resources-3.pp}

\vspace{\baselineskip}\pause

Limitations
\begin{itemize}
\item Purpose of code is not obvious at a first glance
\item Complex resources require complex regular expressions
\end{itemize}
\end{frame}

%%%%%%%%%%%%%%%%%%%%%%%%%%%%%%%%%%%%%%%%%%%%%%%%%%%%%%%%%%%%%%%%%%%%%%%%%%%%%%
\subsection{Puppet 3 -- Example: recursion}

\begin{frame}{\insertsection}{\insertsubsection}
\lstinputlisting{lst/recurse.pp}

\vspace{\baselineskip}\pause

Limitations
\begin{itemize}
\item Needs a way to guarantee unique resource names
\item Error: Somehow looped more than 1000 times \dots
\end{itemize}
\end{frame}

%%%%%%%%%%%%%%%%%%%%%%%%%%%%%%%%%%%%%%%%%%%%%%%%%%%%%%%%%%%%%%%%%%%%%%%%%%%%%%
\subsection{Puppet 4}

\begin{frame}{\insertsection}{\insertsubsection}
Loops in Puppet 4\pause
\begin{itemize}
\item \texttt{each}
\item \texttt{map}
\item \texttt{filter}
\item \texttt{reduce}
\item \texttt{slice}
\end{itemize}

\vspace{\baselineskip}\pause

Also available in Puppet 3.x using \lstinline{parser = future}
\end{frame}

%%%%%%%%%%%%%%%%%%%%%%%%%%%%%%%%%%%%%%%%%%%%%%%%%%%%%%%%%%%%%%%%%%%%%%%%%%%%%%
\section{Loops in Puppet 4}
\subsection{Syntax}

\begin{frame}{\insertsection}{\insertsubsection}

\begin{tikzpicture}%[show background grid]
\node[nopadding]{\lstinputlisting{lst/syntax-1.pp}};
%\fill (0,0) circle (2pt);
\coordinate (para) at (-2.1,-0.32);
\coordinate (item) at (-0.4,-0.32);
\coordinate (code) at (2.0,-0.32);
\end{tikzpicture}

\vspace{\baselineskip}

\begin{columns}[onlytextwidth]
\begin{column}[c]{0.55\textwidth}
\begin{itemize}
\item<2-> \tikz[anchor=base,baseline]%
  \node[nopadding] (ptext) {parameter};
\item<3-> loop variable in
  \tikz[anchor=base,baseline]%
  \node[nopadding] (itext) {\lstinline{|} \dots \lstinline{|}};
\item<4-> code block (\emph{lambda}) in
  \tikz[anchor=base,baseline]%
  \node[nopadding] (ctext) {\lstinline{\{} \dots \lstinline{\}}};
\end{itemize}
\end{column}

\begin{tikzpicture}[overlay]
\draw<2->[->,redmark] (ptext) -- (para);
\draw<3->[->,redmark] (itext) -- (item);
\draw<4->[->,redmark] (ctext) -- (code);
\end{tikzpicture}

\begin{column}<5->[t]{0.38\textwidth}
\tiny
\begin{tabular}{ll}\toprule
  type of parameter & unit of work\\\midrule
  array             & single element\\
  hash              & key-value pair\\
  string            & character\\\bottomrule
\end{tabular}
\end{column}
\end{columns}

\vspace{3\baselineskip}

\visible<6->{
Alternative object-oriented syntax
\lstinputlisting{lst/syntax-2.pp}}
\end{frame}

%%%%%%%%%%%%%%%%%%%%%%%%%%%%%%%%%%%%%%%%%%%%%%%%%%%%%%%%%%%%%%%%%%%%%%%%%%%%%%
\subsection{\texttt{each}}
\begin{frame}{\insertsection}{\insertsubsection}

\begin{block}{\texttt{each}}
Generic loop function
\end{block}

\vspace{\baselineskip}\pause

\begin{itemize}
\item Execute block of code for each item
\item Return original data structure unchanged
\end{itemize}
\end{frame}

%%%%%%%%%%%%%%%%%%%%%%%%%%%%%%%%%%%%%%%%%%%%%%%%%%%%%%%%%%%%%%%%%%%%%%%%%%%%%%
\subsection{\texttt{each} -- Example: Declare resource for each array value}
\begin{frame}{\insertsection}{\insertsubsection}
\lstinputlisting{lst/each-1.pp}
\end{frame}

%%%%%%%%%%%%%%%%%%%%%%%%%%%%%%%%%%%%%%%%%%%%%%%%%%%%%%%%%%%%%%%%%%%%%%%%%%%%%%
\subsection{\texttt{each} -- Example: More program logic in the code block}
\begin{frame}{\insertsection}{\insertsubsection}
\lstinputlisting{lst/each-2.pp}
\end{frame}

%%%%%%%%%%%%%%%%%%%%%%%%%%%%%%%%%%%%%%%%%%%%%%%%%%%%%%%%%%%%%%%%%%%%%%%%%%%%%%
\subsection{\texttt{map}}
\begin{frame}{\insertsection}{\insertsubsection}

\begin{block}{\texttt{map}}
Transform one data structure into another
\end{block}

\vspace{\baselineskip}\pause

\begin{itemize}
\item Execute block of code for each item
\item Use return value of each cycle to build new data structure
\item Return new data structure
\end{itemize}
\end{frame}

%%%%%%%%%%%%%%%%%%%%%%%%%%%%%%%%%%%%%%%%%%%%%%%%%%%%%%%%%%%%%%%%%%%%%%%%%%%%%%
\subsection{\texttt{map} -- Example: Iterate string character by character}
\begin{frame}{\insertsection}{\insertsubsection}
\lstinputlisting[linerange={1-11}]{lst/map-1.pp}\pause
\lstinputlisting[linerange={13}]{lst/map-1.pp}\pause
\lstinputlisting[xleftmargin=2em]{lst/map-1.out}
\end{frame}

%%%%%%%%%%%%%%%%%%%%%%%%%%%%%%%%%%%%%%%%%%%%%%%%%%%%%%%%%%%%%%%%%%%%%%%%%%%%%%
\subsection{\texttt{map} -- Example: Generate array of hashes from an array}
\begin{frame}{\insertsection}{\insertsubsection}
\lstinputlisting{lst/map-2a.pp}
\end{frame}

%%%%%%%%%%%%%%%%%%%%%%%%%%%%%%%%%%%%%%%%%%%%%%%%%%%%%%%%%%%%%%%%%%%%%%%%%%%%%%
\begin{frame}{\insertsection}{\insertsubsection}
\lstinputlisting{lst/map-2.pp}
\end{frame}

%%%%%%%%%%%%%%%%%%%%%%%%%%%%%%%%%%%%%%%%%%%%%%%%%%%%%%%%%%%%%%%%%%%%%%%%%%%%%%
\subsection{\texttt{filter}}
\begin{frame}{\insertsection}{\insertsubsection}

\begin{block}{\texttt{filter}}
Conditionally remove items from arrays and hashes
\end{block}

\vspace{\baselineskip}\pause

\begin{itemize}
\item Execute block of code for each item
\item Return \texttt{true} if item belongs to the result,
  \texttt{false} otherwise
\item Collect wanted items into new data structure
\item Return new data structure
\end{itemize}
\end{frame}

%%%%%%%%%%%%%%%%%%%%%%%%%%%%%%%%%%%%%%%%%%%%%%%%%%%%%%%%%%%%%%%%%%%%%%%%%%%%%%
\subsection{\texttt{filter} -- Example: Filter structured facts}
\begin{frame}{\insertsection}{\insertsubsection}
\lstinputlisting{lst/facts.pp}
\end{frame}

%%%%%%%%%%%%%%%%%%%%%%%%%%%%%%%%%%%%%%%%%%%%%%%%%%%%%%%%%%%%%%%%%%%%%%%%%%%%%%
\begin{frame}{\insertsection}{\insertsubsection}
\lstinputlisting{lst/filter-1.pp}
\end{frame}

%%%%%%%%%%%%%%%%%%%%%%%%%%%%%%%%%%%%%%%%%%%%%%%%%%%%%%%%%%%%%%%%%%%%%%%%%%%%%%
\subsection{\texttt{filter} -- Example: Determine Java JDK to install}
\begin{frame}{\insertsection}{\insertsubsection}
\lstinputlisting[linerange={1-16}]{lst/filter-2.pp}
\end{frame}

\begin{frame}{\insertsection}{\insertsubsection}
\lstinputlisting[linerange={18-34}]{lst/filter-2.pp}
\end{frame}

%%%%%%%%%%%%%%%%%%%%%%%%%%%%%%%%%%%%%%%%%%%%%%%%%%%%%%%%%%%%%%%%%%%%%%%%%%%%%%
\subsection{\texttt{reduce}}

\begin{frame}{\insertsection}{\insertsubsection}

\begin{block}{\texttt{reduce}}
Incremental processing or calculation
\end{block}

\vspace{\baselineskip}\pause

\begin{itemize}
\item Execute block of code for each item
\item Update a memorized result calculated so far
\item Pass memorized result into next cycle
\item Return memorized result from last cycle
\end{itemize}
\end{frame}

%%%%%%%%%%%%%%%%%%%%%%%%%%%%%%%%%%%%%%%%%%%%%%%%%%%%%%%%%%%%%%%%%%%%%%%%%%%%%%
\subsection{\texttt{reduce} -- Example: character filter}

\begin{frame}{\insertsection}{\insertsubsection}
\lstinputlisting{lst/reduce-1.pp}\pause
\lstinputlisting[xleftmargin=2em]{lst/reduce-1.out}
\end{frame}

%%%%%%%%%%%%%%%%%%%%%%%%%%%%%%%%%%%%%%%%%%%%%%%%%%%%%%%%%%%%%%%%%%%%%%%%%%%%%%
\subsection{\texttt{reduce} -- Example: Better validation rules and error messages}

\begin{frame}{\insertsection}{\insertsubsection}
Excerpt from \lstinline{/etc/gai.conf}
\lstinputlisting{lst/gai.conf}

\vspace{\baselineskip}\pause

Implement a validation rule to prevent duplicated values
\end{frame}

%%%%%%%%%%%%%%%%%%%%%%%%%%%%%%%%%%%%%%%%%%%%%%%%%%%%%%%%%%%%%%%%%%%%%%%%%%%%%%
\begin{frame}{\insertsection}{\insertsubsection}
\lstinputlisting{lst/reduce-2.pp}
\end{frame}

%%%%%%%%%%%%%%%%%%%%%%%%%%%%%%%%%%%%%%%%%%%%%%%%%%%%%%%%%%%%%%%%%%%%%%%%%%%%%%
\begin{frame}{\insertsection}{\insertsubsection}
\lstinputlisting{lst/reduce-3.pp}
\end{frame}

%%%%%%%%%%%%%%%%%%%%%%%%%%%%%%%%%%%%%%%%%%%%%%%%%%%%%%%%%%%%%%%%%%%%%%%%%%%%%%
\subsection{\texttt{slice}}

\begin{frame}{\insertsection}{\insertsubsection}

\begin{block}{\texttt{slice}}
Divide an array into smaller pieces and process each part
\end{block}

\vspace{\baselineskip}\pause

\begin{itemize}
\item Execute block of code for groups of $n$ items
%\item Return memorized result from last cycle
\end{itemize}
\end{frame}

%%%%%%%%%%%%%%%%%%%%%%%%%%%%%%%%%%%%%%%%%%%%%%%%%%%%%%%%%%%%%%%%%%%%%%%%%%%%%%
\subsection{\texttt{slice} -- Example: Iterate array in pairs}

\begin{frame}{\insertsection}{\insertsubsection}
\lstinputlisting{lst/slice-1.pp}\pause
\lstinputlisting[xleftmargin=2em]{lst/slice-1.out}

\vspace{\baselineskip}\pause

Unfortunately no \textit{real life}\texttrademark{} example \dots
\end{frame}

%%%%%%%%%%%%%%%%%%%%%%%%%%%%%%%%%%%%%%%%%%%%%%%%%%%%%%%%%%%%%%%%%%%%%%%%%%%%%%
\section{Wrap up}

\begin{frame}{\insertsection}{\insertsubsection}

Puppet 4 adds five powerful iteration functions

\vspace{\baselineskip}\pause

Use cases: process arrays/hashes/strings to
\begin{itemize}
\item individually manage each item\pause
\item transform the data structure\pause
\item select items based on conditions\pause
\item create validations with better error messages
%\item Implement user functions in Puppet instead of Ruby
\end{itemize}

\vspace{\baselineskip}\pause

Arrays and hashes can be
\begin{itemize}
\item class/type parameters
\item structured facts
\end{itemize}
\end{frame}

%%%%%%%%%%%%%%%%%%%%%%%%%%%%%%%%%%%%%%%%%%%%%%%%%%%%%%%%%%%%%%%%%%%%%%%%%%%%%%
\section{Contact}
\begin{frame}{\insertsection}{\strut}

%Icons from: https://www.iconfinder.com/search/?q=iconset:simple-icons
\begin{tabular}{m{40pt}l}
\iconentry{email}{\mailto{stm@kill-9.net}}
\iconentry{twitter}{\nolinkurl{@UnixMagus}}
\iconentry{blogger}{\url{http://www.moeding.net}}
\iconentry{github}{\url{https://github.com/smoeding}}
\iconentry{puppetforge}{\url{https://forge.puppetlabs.com/stm}}
\end{tabular}
\footnote{\tiny Icons by Dan Leech, \url{http://simpleicons.org}}
\end{frame}

%%%%%%%%%%%%%%%%%%%%%%%%%%%%%%%%%%%%%%%%%%%%%%%%%%%%%%%%%%%%%%%%%%%%%%%%%%%%%%
%\appendix
%\section{\appendixname}
%\begin{frame}{\appendixname}{\insertsection}
%\end{frame}

\end{document}
